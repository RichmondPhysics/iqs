
\section{Conservation of Angular Momentum}

Name \rule{2.0in}{0.1pt}\hfill{}Section \rule{1.0in}{0.1pt}\hfill{}Date \rule{1.0in}{0.1pt}

\textbf{Objectives} 

To test the Law of Conservation of Angular Momentum and to explore the applicability
of angular momentum conservation among objects that experience no external torques. 

\textbf{Apparatus}

\begin{itemize}
\item A Rotating Disk System 
\item A mass of 1 kg 
\item A meter stick and a ruler 
\item A small water bubble level
\item A video analysis system
\end{itemize}
\textbf{Overview }

As a consequence of Newton's laws, angular momentum like linear momentum is
believed to be conserved in isolated systems. This means that, no matter how
many internal interactions occur, the total angular momentum of any system should remain constant if there are no external torques. When one of the objects gains some angular momentum another part of the system must lose the same amount. If angular momentum isn't conserved, then we believe that there is some outside torque acting on the system. By expanding the boundary of the system to include the source of that torque we can always preserve the Law of Angular Momentum Conservation. 

In this unit you will test the notion of the conservation of angular momentum.
As in the test of the conservation of linear momentum, we will investigate what
happens when two bodies undergo a ``rotational'' collision.
You will drop a large weight onto a rotating disk and determine the moment of
inertia, the angular speed, and finally, the angular momentum of the rotator-disk-weight
system before and after this perfectly inelastic collision.

\textbf{Activity  \stepcounter{activity}\arabic{activity}: The Moment of Inertia Before and After the Collision}

(a) Calculate the theoretical value of the moment of inertia of the metal disk
using basic measurements of its radius and mass. Be sure to state units and
show the expression you used!
\vspace{5mm}

\( r_{d} =\)  \hfill{}\( M_{d}= \) \hfill{}
\vspace{5mm}

\( I_{d}= \)
\vspace{5mm}

(b) The rotating fixture that holds the disk has a complex shape. We have determined its moment of inertia without the disk and recorded the result. Record that value here. Be sure to state units.
\vspace{5mm}

\( I_{f} =\)
\vspace{5mm}

(c) After dropping the weight on the rotating disk, the system will have a new
moment of inertia. Derive a formula for the moment of inertia of a cylindrical-shaped weight of mass \( m_{w} \) and radius \( r_{w} \) revolving about the origin at a distance \( r_{r} \). (You will have to use the parallel axis theorem to do this.)
\vspace{5mm}

\( I_{w} =\)  
\vspace{5mm}

(d) Measure the mass of the weight and use a vernier caliper to measure its diameter.
\vspace{5mm}

\( m_{w} =\)  \hfill{}\(r_{w} =\) \hfill{}
\vspace{5mm}

(e) Come up with a formula for the moment of inertia $I$ of the whole system
before and after the collision and calculate the moment of inertia before the
collision only. (The moment of inertia after the collision will be determined AFTER you do the experiment, because you will not know \( r_{r} \) until after you do the experiment.) Don't forget to include the units in \( I_{before} \).
\vspace{5mm}

\( I_{\mbox{\small before}}= \) 
\vspace{5mm}

\( I_{\mbox{\small after}} =\)  
\vspace{5mm}

\textbf{Activity  \stepcounter{activity}\arabic{activity}: Making a Movie of the Collision} 

(a) Place the video camera about 1 m above the rotator, align the camera with
the center of the rotator using the pendulum, and center the rotator in the
field of view of the camera by viewing it with the video capture software.
Place a ruler of known length in the field of view of the camera and parallel
to one side of the frame. Check that the rotator is flat with the small water-bubble level.

(b) Give the rotator a push and begin recording its motion with the video camera. 
See \textbf{Appendix D: Video Analysis} for details on making the movie. While the rotator is moving
hold the 1-kg weight near the rim of the metal disk and close to, but not quite
touching, the surface of the moving metal disk. After at least one revolution
of the metal disk drop the 1-kg mass onto the disk and record the motion of
the disk for at least one revolution afterward. 

(c) \textit{Before removing the 1 kg weight from the metal disk}, 
determine the distance of the center of the weight you dropped from the center of the rotator \( r_{r} \). 
To do this, measure the distance from the center
of the rotator to the edge of the weight \( r_{edge} \) and use the result from Activity 1 part (d) 
for the radius of the weight \( r_{w} \). 
Calculate the distance from the origin to the center of the weight \( r_{r} \) (it is just the 
sum of \( r_{edge} \) and \( r_{w} \)). 
Use these results and those from Activity 1 part (e) to calculate the final moment of inertia.
\vspace{5mm}

\( r_{\mbox{\small edge}}  =\) \hfill{}\( r_{w} \) = \hfill{}\( r_{r} \) =\hfill{}
\vspace{5mm}

\( I_{\mbox{\small after}} =\)  
\vspace{5mm}

\textbf{Activity  \stepcounter{activity}\arabic{activity}: Measurement of Angular Velocity}

Determine the angular speed before and after the collision.

(a) Determine the angular displacement of the rotator as a function of time before the weight was dropped. 
To do this task follow the instructions in \textbf{Appendix D: Video Analysis} for recording, calibrating, and analyzing a movie data file. 
\textbf{Important:} Be careful to place the origin of your coordinate system on the axle of the
rotator so the angular displacement you measure will be the desired one. 
Use a marker at or near the edge of the disk to record position for each frame. 
The resulting file should contain three columns with the values of time, $x$-position, and $y$-position.
Calculate the angular position for each time using the $x-$ and $y-$position data, fit the results,
and extract the angular speed of the rotator before the `collision'.

\( \omega _{\mbox{\small before}}=\)

\vspace{5mm}

\newpage

(b) Now follow a similar procedure to determine the angular speed after the collision.
Fit the data and extract the angular speed of the rotator after the `collision'.

\( \omega _{\mbox{\small after}}=\)

\vspace{20mm}

\textbf{Activity  \stepcounter{activity}\arabic{activity}: Calculation of Angular Momentum}

(a) Calculate the angular momentum before and after the collision, including UNITS. Calculate the difference between the two results and record it below. 
Go around to the other lab groups and get their results for the difference between the angular momenta before and after the collision.
Make a histogram of the results you collect and calculate the average and standard deviation.
For information on making histograms, see \textbf{Appendix C}. For information on calculating the average and
standard deviation, see \textbf{Appendix A}. Record the average and standard deviation here.
Attach the histogram to this unit.
\vspace{5mm}

\( L_{\mbox{\small before}}= \)  
\vspace{5mm}

\( L_{\mbox{\small after}}= \)
\vspace{5mm}

\( \Delta L= L_{\mbox{\small after}} - L_{\mbox{\small before}} = \)  
\vspace{30mm}

(b) What is your expectation for the difference between the initial and final angular momentum?
Do the data from the class support this expectation? 
Use the average and standard deviation for the class to quantitatively answer this question.
\vspace{20mm}


(c) Within the limits of experimental uncertainty, is angular momentum 
conserved?  Be quantitative.
\vspace{20mm}

(c) Would the procedure you followed above change if the weight was moving horizontally at a constant velocity when you dropped it? 
If it changed, what would be different?
\vspace{20mm}

