\appendix


\section{Treatment of Experimental Data}

\textbf{Recording Data }

When performing an experiment, record all required original observations as
soon as they are made. By ``original observations'' is meant
what you actually see, not quantities found by calculation. For example, suppose
you want to know the stretch of a coiled spring as caused by an added weight.
You must read a scale both before and after the weight is added and then subtract
one reading from the other to get the desired result. The proper scientific
procedure is to record both readings as seen. Errors in calculations can be
checked only if the original readings are on record.

All data should be recorded with units. If several measurements are made of
the same physical quantity, the data should be recorded in a table with the
units reported in the column heading.

\textbf{Significant Figures} 

A laboratory worker must learn to determine how many figures in any measurement
or calculation are reliable, or ``significant'' (that is, have
physical meaning), and should avoid making long calculations using figures which
he/she could not possibly claim to know. \textit{All sure figures plus one estimated
figure are considered significant.}

The measured diameter of a circle, for example, might be recorded to four significant
figures, the fourth figure being in doubt, since it is an estimated fraction
of the smallest division on the measuring apparatus. How this doubtful fourth
figure affects the accuracy of the computed area can be seen from the following
example.

Assume for example that the diameter of the circle has been measured as .526\underbar{4} 
cm, with the last digit being in doubt as indicated by the line under it. When
this number is squared the result will contain eight digits, of which the last
five are doubtful. Only one of the five doubtful digits should be retained,
yielding a four-digit number as the final result.

In the sample calculation shown below, each doubtful figure has a short line
under it. Of course, each figure obtained from the use of a doubtful figure
will itself be doubtful. The result of this calculation should be recorded as
0.2771 cm\( ^{2} \), including the doubtful fourth figure. (The zero to the
left of the decimal point is often used to emphasize that no significant figures
precede the decimal point. This zero is not itself a significant figure.)

{\par\centering (.526\underbar{4} cm)\( ^{2} \) = .277\underbar{09696} cm\( ^{2} \)
= 0.277\underbar{1} cm\( ^{2} \)\par}

\textit{In multiplication and division, the rule is that a calculated result
should contain the same number of significant figures as the least that were
used in the calculation.}

\textit{In addition and subtraction, do not carry a result beyond the first
column that contains a doubtful figure.}

\textbf{Statistical Analysis} 

Any measurement is an intelligent estimation of the true value of the quantity
being measured. To arrive at a ``best value'' we usually make
several measurements of the same quantity and then analyze these measurements
statistically. The results of such an analysis can be represented in several
ways. Those in which we are most interested in this course are the following:

\underbar{Mean} - The mean is the sum of a number of measurements of a quantity
divided by the number of such measurements.
(In other words, the mean is the same thing as what people
generally call the ``average.'')
It generally represents the best estimate
of true value of the measured quantity. 

\underbar{Standard Deviation} - The standard deviation (\( \sigma  \)) is a
measure of the range on either side of the mean within which approximately two-thirds
of the measured values fall. For example, if the mean is 9.75 m/s\( ^{2} \)
and the standard deviation is 0.10 m/s\( ^{2} \), then approximately two-thirds
of the measured values lie within the range 9.65 m/s\( ^{2} \) to 9.85 m/s\( ^{2} \).
A customary way of expressing an experimentally determined value is: Mean\( \pm  \)\( \sigma  \),
or (9.75\( \pm  \) 0.10) m/s\( ^{2} \). Thus, the standard deviation is an
indicator of the spread in the individual measurements, and a small \( \sigma  \)
implies high precision. Also, it means that the probability of any future measurement
falling in this range is approximately two to one. The equation for calculating
the standard deviation is
\[
\sigma =\sqrt{\frac{\sum \left( x_{i}-\left\langle x\right\rangle \right) ^{2}}{N-1}}\]
 where \( x_{i} \) are the individual measurements, \( \left\langle x\right\rangle  \)
is the mean, and $N$ is the total number of measurements.

\underbar{\% Difference} - Often one wishes to compare the value of a quantity
determined in the laboratory with the best known or ``accepted value''
of the quantity obtained through repeated determinations by a number of investigators.
\textit{The \% difference is calculated by subtracting the accepted value from
your value, dividing by the accepted value, and multiplying by 100.} If your
value is greater than the accepted value, the \% difference will be positive.
If your value is less than the accepted value, the \% difference will be negative.
The \% difference between two values in a case where neither is an accepted
value can be calculated by choosing either one as the accepted value.

